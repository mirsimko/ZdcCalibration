\documentclass[a4paper,10pt]{article}
\usepackage[utf8]{inputenc}
\usepackage{amsmath}
\usepackage{amsfonts}
\usepackage{breakurl}
\usepackage{indentfirst}
\usepackage{graphicx}
\usepackage{enumerate}
\usepackage[breaklinks, hidelinks]{hyperref}
\usepackage{url}
\usepackage{booktabs}

\newcommand{\vect}[1]{\ensuremath{\boldsymbol{#1}}}
\newcommand{\ddd}{\ensuremath{\,\mathrm{d}}}
\newcommand{\dd}{\mathrm{d}}
\newcommand{\Var}{\mathrm{Var}}
\newcommand{\pravYi}{P(Y_i|\alpha, \beta, \epsilon_i)}
\newcommand{\pravY}{P(\boldsymbol{Y}|\alpha, \beta, \epsilon_i)}
\newcommand{\nsigma}{\frac{N}{\sigma^2}}
\newcommand{\arpxsq}{\overline{(X^2)}}
\newcommand{\sigman}{\frac{\sigma^2}{N}}
\newcommand{\jmenovatel}{ \arpxsq - \overline{X}^2}
\usepackage{amsmath}
%opening
\title{ZDC voltages}
\author{}
\date{}

\begin{document}

\maketitle

The voltages were calculated according to the formula
$$
G = aU^b
$$
where $G$ is a gain and $U$ stands for voltage. The coefficients $a$ and $b$ are 
extracted from the following document:\\
\url{https://drupal.star.bnl.gov/STAR/system/files/ZDC_Document_20101005.pdf}\\
and are equal to approximately
\begin{equation}
b=4.2\,, \qquad a=4.0\,.
\end{equation}

We take into account that the desired position of the single neutron peak is at 60 ADC values
and the desired ratio between the towers is 6:3:1. To calculate the desired voltages we use the
formula
\begin{equation}
U_{\text{result}} = U_\text{current}\left(\frac{G_\text{desired}}{G} 
\frac{R_\text{desired}}{R}\right)^{1/4.2}
\end{equation}



where $G$ is the current position of the neutron peak, $G_\text{desired}$ is the desired position
of the neutron peak (currently 60), $R_{desired}$ is the desired ratio between the gain of the ADC SUM
tower and the current tower, and $R$ is the current ratio.
The resulting voltages are in Table \ref{uncorected}. The position of the single neutron peak and
ratios between the towers with the resulting voltages from the test are shown in Table \ref{corected}.
New high-voltage values are also calculated.

\begin{table}[!htb] 
\caption{Calculated voltages from the position of the single neutron peak and ratios 
between ZDC towers}
\label{uncorected}
\begin{center}
\begin{tabular}{lccccc}
 \toprule
 &$U_\text{current}$[V]&Single n pos.&gain ratio&desired gain ratio&$U_\text{result}$[V]\\
\midrule
 East&2471&38.33&0.689&0.6&2661\\
     &2779&38.33&0.228&0.3&3300\\
     &2353&38.33&0.083&0.1&2735\\
 \midrule
West&2532&43.56&0.665&0.6&2667\\
    &2643&43.56&0.278&0.3&2905\\
    &2671&43.56&0.057&0.1&3294\\
 \bottomrule
\end{tabular}
\end{center}
\end{table}

\begin{figure}[!htb]
\begin{center}
\includegraphics[width=.8\textwidth]{CorrectedEast.pdf}
\end{center}
\caption{Single neutron peak for east towers.}
\end{figure}

\begin{figure}[!htb]
\begin{center}
\includegraphics[width=.8\textwidth]{CorrectedWest.pdf}
\end{center}
\caption{Single neutron peak for west towers.}
\end{figure}

\begin{table}[!htb] 
\caption{Calculated voltages from the test with corrected high voltages}
\label{corected}
\begin{center}
\begin{tabular}{lccccc}
 \toprule
 &$U_\text{current}$[V]&Single n pos.&gain ratio&desired gain ratio&$U_\text{result}$[V]\\
\midrule
 East&2547&53.32&0.569&0.6&2652\\
     &3159&53.32&0.330&0.3&3176\\
     &2618&53.32&0.101&0.1&2689\\
 \midrule
West&2333&31.07&0.563&0.6&2771\\
    &2542&31.07&0.319&0.3&2931\\
    &2883&31.07&0.118&0.1&3239\\
 \bottomrule
\end{tabular}
\end{center}
\end{table}



\end{document}

