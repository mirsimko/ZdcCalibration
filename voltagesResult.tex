\documentclass[a4paper,10pt]{article}
\usepackage[utf8]{inputenc}
\usepackage{amsmath}
\usepackage{amsfonts}
\usepackage{breakurl}
\usepackage{indentfirst}
\usepackage{graphicx}
\usepackage{enumerate}
\usepackage[breaklinks, hidelinks]{hyperref}
\usepackage{url}
\usepackage{booktabs}

\newcommand{\vect}[1]{\ensuremath{\boldsymbol{#1}}}
\newcommand{\ddd}{\ensuremath{\,\mathrm{d}}}
\newcommand{\dd}{\mathrm{d}}
\newcommand{\Var}{\mathrm{Var}}
\newcommand{\pravYi}{P(Y_i|\alpha, \beta, \epsilon_i)}
\newcommand{\pravY}{P(\boldsymbol{Y}|\alpha, \beta, \epsilon_i)}
\newcommand{\nsigma}{\frac{N}{\sigma^2}}
\newcommand{\arpxsq}{\overline{(X^2)}}
\newcommand{\sigman}{\frac{\sigma^2}{N}}
\newcommand{\jmenovatel}{ \arpxsq - \overline{X}^2}
\usepackage{amsmath}
%opening
\title{ZDC voltages}
\author{}
\date{}

\begin{document}

\maketitle

The voltages were calculated according to the formula
$$
G = aU^b
$$
where $G$ is a gain and $U$ stands for voltage. The coefficients $a$ and $b$ are 
extracted from the following document:\\
\url{https://drupal.star.bnl.gov/STAR/system/files/ZDC_Document_20101005.pdf}\\
and are equal to approximately
\begin{equation}
b=4.2\,, \qquad a=4.0\,.
\end{equation}

We take into account that the desired position of the single neutron peak is at 60 ADC values
and the desired ratio between the towers is 6:3:1. To calculate the desired voltages we use the
formula
\begin{equation}
U_{\text{result}} = U_\text{current}\left(\frac{G_\text{desired}}{G} 
\frac{R_\text{desired}}{R}\right)^{1/4.2}
\end{equation}
where $G$ is the current position of the neutron peak, $G_\text{desired}$ is the desired position
of the neutron peak (currently 60), $R_{desired}$ is the desired ratio between the gain of the ADC SUM
tower and the current tower, and $R$ is the current ratio.
The resulting voltages are in Table \ref{uncorected}. The position of the single
neutron peaks in this test
are shown in Figures \ref{eastPlot} and \ref{westPlot}.
The ratios between the towers with the resulting voltages from the test
are shown in Table \ref{corected}.
Subsequently, new high-voltage values are also calculated.

\begin{table}[htb] 
\caption{Calculated voltages from the position of the single neutron peak and ratios 
between ZDC towers}
\label{uncorected}
\begin{center}
\begin{tabular}{lccccc}
 \toprule
 &$U_\text{current}$[V]&Single n pos.&gain ratio&desired gain ratio&$U_\text{result}$[V]\\
\midrule
 East&2471&38.33&0.689&0.6&2661\\
     &2779&38.33&0.228&0.3&3300\\
     &2353&38.33&0.083&0.1&2735\\
 \midrule
West&2532&43.56&0.665&0.6&2667\\
    &2643&43.56&0.278&0.3&2905\\
    &2671&43.56&0.057&0.1&3294\\
 \bottomrule
\end{tabular}
\end{center}
\end{table}

\begin{figure}[htb]
\begin{center}
\includegraphics[width=.8\textwidth]{CorrectedEast.pdf}
\end{center}
\caption{Single neutron peak for east towers. Resulting voltages from Table \ref{uncorected}.}
\label{eastPlot}
\end{figure}

\begin{figure}[htb]
\begin{center}
\includegraphics[width=.8\textwidth]{CorrectedWest.pdf}
\end{center}
\caption{Single neutron peak for west towers. Resulting voltages from Table \ref{uncorected}.}
\label{westPlot}
\end{figure}

\begin{table}[htb] 
\caption{Calculated voltages from the test with corrected high voltages.
The value 50 was set as the deseired position of the single neutron peak}
\label{corected}
\begin{center}
\begin{tabular}{lccccc}
 \toprule
 &$U_\text{current}$[V]&Single n pos.&gain ratio&desired gain ratio&$U_\text{result}$[V]\\
\midrule
 East&2547&53.32&0.569&0.6&2540\\
     &3159&53.32&0.330&0.3&3041\\
     &2618&53.32&0.101&0.1&2575\\
 \midrule
West&2333&31.07&0.563&0.6&2653\\
    &2542&31.07&0.319&0.3&2806\\
    &2883&31.07&0.118&0.1&3101\\
 \bottomrule
\end{tabular}
\end{center}
\end{table}

Channels where the voltage would exceed 3000 V in table \ref{corected} were set to 3000 V to
protect the PMTs. New single neutron peak as well as the double neutron peak were found and
plotted in Figures \ref{eastThird}
and \ref{westThird}. The position of the double neutron peak was set as twice the mean
of the single neutron peak.
The resulting voltages in the next iteration are in table \ref{thirdCalib}.

\begin{figure}[htb]
\begin{center}
\includegraphics[width=.8\textwidth]{neutronPeaksEastFinal.pdf}
\end{center}
\caption{Single neutron and double neutron peaks for east towers.
Voltages were taken from table~\ref{corected}.}
\label{eastThird}
\end{figure}

\begin{figure}[htb]
\begin{center}
\includegraphics[width=.8\textwidth]{neutronPeaksWestFinal.pdf}
\end{center}
\caption{Single neutron and double neutron peaks for west towers.
Voltages were taken from table~\ref{corected}.}
\label{westThird}
\end{figure}

\begin{table}[htb] 
\caption{Calculated voltages from the test with corrected high voltages from table~\ref{corected}.
The value 58 was set as the desired position of the single neutron peak}
\label{thirdCalib}
\begin{center}
\begin{tabular}{lccccc}
 \toprule
 &$U_\text{current}$[V]&Single n pos.&gain ratio&desired gain ratio&$U_\text{result}$[V]\\
\midrule
East  &2540  &58 &0.623  &0.6 &2517 \\
      &3000  &58 &0.274  &0.3 &3066 \\
      &2575  &58 &0.103	&0.1 &2557 \\
\midrule
West  &2653  &64 &0.634 &0.6 &2558 \\
      &2806  &64 &0.297 &0.3 &2748 \\
      &3000  &64 &0.069 &0.1 &3196 \\
\bottomrule
\end{tabular}
\end{center}
\end{table}

The results of the test with voltages from Table \ref{thirdCalib} are shown in
Figures \ref{oneMoreTestEast} and \ref{oneMoreTestWest}\@. The single neutron and double
neutron peaks were fitted. The ratios between the towers in the East were measured as
60.29:29.41:10.30 and in the West 61.07:30.52:8.41. 

In conclusion, the ratios between the towers were closer to the desired ones in the
test with voltages from Table \ref{thirdCalib}. The ratios between the single and double
neutron peak in the East towers were closer to the real values in the test from
Table \ref{corected}. Therefore, we suggest that the result voltages from Table \ref{corected}
are used for the East towers and the voltages from Table \ref{thirdCalib} are used for the
West towers. This setup was tested and the neutron peaks are plotted in Figures \ref{finalTestEast}
and \ref{finalTestWest}. The ratios between the towers were calculated as
61.30:28.58:10.17 for the East towers and 61.02:30.53:8.45 for the West towers.
Which is close enough to the desired values.

\begin{figure}[!htb]
\begin{center}
\includegraphics[width=.8\textwidth]{oneMoreTestEast.pdf}
\end{center}
\caption{Single neutron and double neutron peaks for east towers.
Voltages were taken from table~\ref{thirdCalib}.}
\label{oneMoreTestEast}
\end{figure}

\begin{figure}[!htb]
\begin{center}
\includegraphics[width=.8\textwidth]{oneMoreTestWest.pdf}
\end{center}
\caption{Single neutron and double neutron peaks for west towers. 
Voltages were taken from table~\ref{thirdCalib}.}
\label{oneMoreTestWest}
\end{figure}

\begin{figure}[!htb]
\begin{center}
\includegraphics[width=.8\textwidth]{FinalTestEast.pdf}
\end{center}
\caption{Single neutron and double neutron peaks for east towers in the final test. Voltages were
for the East towers taken from table~\ref{corected}.}
\label{finalTestEast}
\end{figure}

\begin{figure}[!htb]
\begin{center}
\includegraphics[width=.8\textwidth]{FinalTestWest.pdf}
\end{center}
\caption{Single neutron and double neutron peaks for west towersin the final test. Voltages
for the West towers were taken from table~\ref{thirdCalib}.}
\label{finalTestWest}
\end{figure}
\end{document}